% Guide to the literature
\section{Background}  \label{sec:background}

[\emph{It is not yet clear how to organize the subsections, which currently overlap
  significantly.}]


%%%%%%%%%%%%%%%%%%%%%%%%%%%%%%%%%%%%%%%%%%%%%%%%%%%%%%%%%%%%%%%%%%%%%%%%%%%%%%%%%%

\subsection{Basics of Bayesian statistics}  \label{subsec:basic_Bayes}

\bi
  \I Not an exhaustive recounting, but the minimum needed to make sense of
    the review.

  \I Pointers to the literature (which is listed later).
\ei

%%%%%%%%%%%%%%%%%%%%%%%%%%%%%%%%%%%%%%%%%%%%%%%%%%%%%%%%%%%%%%%%%%%%%%%%%%%%%%%%%%

\subsection{Conceptual or philosophical issues} \label{subsec:conceptual_issues}

\bi
  \I In comparison to frequentist approaches, the emphasis is on sampling of posteriors
    rather than optimization (e.g., finding the maximum of the likelihood).

  \I 
One of the essential, philosophical, difference between Bayesian and frequentist approach is the emphasis put on prior knowledge. Consider the textbook example of coin tossing: 
{\it If} the toss is perfectly random and the coin is perfect, {\it then} the probability $P$ of next head or tail is 0.5 {\it irrespective of the history} of all previous tosses. For example, there could already have been one million tosses that all gave head: in a strict frequentist approach, the probability of observing head in the next toss would remain 0.5 exactly. {\color{blue} I suppose we could introduce bias in our statistical model}. By contrast, Bayesian statistics provides a very natural framework to relax this strict interpretation. In our example, putting a (non-uniform) prior distribution to reflect the observation of these one millions heads implies that we accept that the coin may not be perfect and/or that the tosses were not truly random. As a consequence, the probability of observing a next head can be $\neq 0.5$.

  \I 
In a scientific context, the notion of prior knowledge makes the Bayesian approach very pertinent, since there is almost always something we know about our model/experiment

  \I Younger generation statisticians do not have the baggage of the frequentist-Bayesian wars;
    they freely use both approaches as tools to do statistical analysis. 

  \I Is exchangability a purely formal aspect or an important consideration for practical
    purposes? 
\ei

%%%%%%%%%%%%%%%%%%%%%%%%%%%%%%%%%%%%%%%%%%%%%%%%%%%%%%%%%%%%%%%%%%%%%%%%%%%%%%%%%%

