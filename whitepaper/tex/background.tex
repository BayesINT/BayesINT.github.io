% Guide to the literature
\section{Background}  \label{sec:background}

[\emph{It is not yet clear how to organize the subsections, which currently overlap
  significantly.}]


%%%%%%%%%%%%%%%%%%%%%%%%%%%%%%%%%%%%%%%%%%%%%%%%%%%%%%%%%%%%%%%%%%%%%%%%%%%%%%%%%%

\subsection{Basics of Bayesian statistics}  \label{subsec:basic_Bayes}

\bi
  \I Not an exhaustive recounting, but the minimum needed to make sense of
    the review.

  \I Pointers to the literature (which is listed later).
\ei

%%%%%%%%%%%%%%%%%%%%%%%%%%%%%%%%%%%%%%%%%%%%%%%%%%%%%%%%%%%%%%%%%%%%%%%%%%%%%%%%%%

\subsection{Conceptual or philosophical issues} \label{subsec:conceptual_issues}

\bi
  \I In comparison to frequentist approaches, the emphasis is on sampling of posteriors
    rather than optimization (e.g., finding the maximum of the likelihood).

  \I Philosophically, emphasis is put on prior knowledge. Textbook example of coin flipping: 
     {\it If} the coin is perfect, then probability of next head or tail is 0.5 
     {\it irrespective of history}. Contrariwise, putting a (non-uniform) prior 
     in Bayesian framework implies that probability can be $\neq 0.5$

  \I Remark: in a scientific context, the notion of prior knowledge makes Bayesian approch
     very pertinent, since there is almost always something we know about our model/experiment
    
  \I Younger generation statisticians do not have the baggage of the frequentist-Bayesian wars;
    they freely use both approaches as tools to do statistical analysis. 

  \I Is exchangability a purely formal aspect or an important consideration for practical
    purposes? 
\ei

%%%%%%%%%%%%%%%%%%%%%%%%%%%%%%%%%%%%%%%%%%%%%%%%%%%%%%%%%%%%%%%%%%%%%%%%%%%%%%%%%%

