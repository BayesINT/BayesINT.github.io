% Guide to the literature
\section{Background}  \label{sec:background}

[\emph{It is not yet clear how to organize the subsections, which currently overlap
  significantly.}]

%%%%%%%%%%%%%%%%%%%%%%%%%%%%%%%%%%%%%%%%%%%%%%%%%%%%%%%%%%%%%%%%%%%%%%%%%%%%%%%%%%

\subsection{Basics of Bayesian statistics}  \label{subsec:basic_Bayes}

\bi
  \I Not an exhaustive recounting, but the minimum needed to make sense of
    the review.
  \I Pointers to the literature (which is listed later).
\ei

%%%%%%%%%%%%%%%%%%%%%%%%%%%%%%%%%%%%%%%%%%%%%%%%%%%%%%%%%%%%%%%%%%%%%%%%%%%%%%%%%%

\subsection{Conceptual or philosophical issues} \label{subsec:conceptual_issues}

\bi
  \I In comparison to frequentist approaches, the emphasis is on sampling of posteriors
    rather than optimization (e.g., finding the maximum of the likelihood).

  \I Younger generation statistians do not have the baggage of the frequentist-Bayesian wars;
    they freely use both approaches as tools to do statistical analysis. 
\ei

%%%%%%%%%%%%%%%%%%%%%%%%%%%%%%%%%%%%%%%%%%%%%%%%%%%%%%%%%%%%%%%%%%%%%%%%%%%%%%%%%%

