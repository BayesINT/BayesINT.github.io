\section{Introduction}
\label{Sec:Introduction}

QCD theory and modeling, benefitting from continuous experimental guidance, have led to the development of a standard model to describe the dynamic space-time evolution of the Little Bangs created in high energy nuclear collisions\cite{Heinz:2013wva}. Collective behavior observed via correlations among the particles produced in the debris of these explosions led to the discovery that, soon after the initial collision, dense QCD matter thermalizes with a very high initial temperature and forms a strongly coupled quark-gluon plasma (QGP). Surprisingly, this matter that filled the early universe turns out to be a liquid with a specific viscosity (the ratio of viscosity to entropy density)  $\eta/s$ smaller than that of any other known substance\cite{Csernai:2006zz,Gale:2012rq} and very near a limiting value for this quantity that is characteristic of plasmas in infinitely strongly interacting gauge theories with a dual gravitational description\cite{Kovtun:2004de}.

Despite continued progress, estimates of the $\eta/s$ of QGP generally remain upper bounds, due to systematic uncertainties arising from an incomplete knowledge of the initial state. However, the unanticipated recent discovery that ripples in the near-perfect QGP liquid bring information about nucleonic and sub- nucleonic gluon fluctuations in the initial state into the final state\cite{Gale:2012in} has opened new possibilities to study the dense gluon fields and their quantum fluctuations in the colliding nuclei via correlations between final state particles. Mapping the transverse and longitudinal dependence of the initial gluon fluctuation spectrum will provide both a test for QCD calculations in a high gluon density regime as well as the description of the initial state necessary to further improve the determination of $\eta/s$.

Understanding strongly coupled or strongly correlated systems is at the intellectual forefront of multiple areas of physics. One example is the physics of ultra-cold fermionic atoms, where application of a magnetic field excites a strong resonance. In an atomic trap, such atoms form a degenerate Fermi liquid, which can be studied in exquisite detail\cite{O'Hara:2002zz}. At temperatures below $\sim 0.1\ \mu\mathrm{K}$ the atoms interact via a Feshbach resonance to form a superfluid\cite{Kinast:2004zza}. Strongly correlated electron systems in condensed matter provide another strongly coupled system\cite{Rameau:2014gma}. Here, the elementary interaction is not strong, but its role is amplified by the large number of interacting particles and their ability to dynamically correlate their quantum wave functions. In conventional plasma physics strong coupling is observed in warm, dense matter and dusty plasmas\cite{Chan:2004} residing in astrophysical environments, such as the rings of Saturn, as well as in thermonuclear fusion.

A unique feature of QCD matter compared to the other strongly coupled systems is that the interaction is specified directly at the Lagrangian level by a fundamental theory. Thus we have a chance to understand how a strongly coupled fluid emerges from a microscopic theory that is precisely known. The high temperature achieved in nuclear collisions, combined with the precision achieved by the numerical solutions of lattice gauge theory, permits {\em ab initio} calculation of equilibrium properties of hot QCD matter without any model assumptions or approximations.

At the same time, the discovery of strongly coupled QGP poses many questions. How do its properties vary over a broad range of temperature and chemical potential?
Is there a critical point in the QCD phase diagram where the hadron gas to QGP phase transition becomes first-order?  What is the smallest droplet of hot QCD matter whose behavior is liquid-like? What are the initial conditions that lead to hydrodynamic behavior, and can they be extracted from the experimental data? Can the underlying degrees of freedom in the liquid be resolved with jets and heavy flavor probes? An understanding of how the properties of the liquid emerge requires probing the matter at varied, more microscopic, length scales than those studied to date, as well as examining its production in systems of different size over a range of energies.

Answering these and other questions will depend also on an intensive modeling and computational effort to
simultaneously determine the set of key parameters needed for a multi-scale characterization of the QGP medium as well as the initial state from which it emerges. This phenomenological effort requires broad experimental input from a diverse set of measurements, including but not limited to
\begin{enumerate}

\item A more extensive set of heavy quark measurements to determine the diffusion coefficient of heavy quarks.

\item Energy scans to map the phase diagram of QCD and the dependence of transport coefficients on the temperature and chemical potential.

\item  Collisions of nuclei with varied sizes, including p+A and very high multiplicity p+p collisions, to study the emergence of collective phenomena and parton interactions with the nuclear medium.

\item The quantitative characterization of the electromagnetic radiation emitted by the Little Bangs and its spectral anisotropies.

\item A search for chiral symmetry restoration through measurements of lepton pair masses.

\item Detailed investigation of medium effects on the production rates and internal structure of jets of hadrons, for multi-scale tomographic studies of the medium. 
\end{enumerate}
This program will illuminate how ``more'' becomes ``different'' in matter governed by the equations of QCD.

RHIC and the LHC, together provide an unprecedented opportunity to pursue the program listed above and to thereby resolve the open questions in thermal properties of QCD matter. While collisions at the LHC create temperatures well above those needed for the creation of QGP and may thus be able to explore the expected transition from a strongly coupled liquid to a weakly coupled gaseous phase at higher temperatures, the RHIC program uniquely enables  research at temperatures close to the phase transition where the coupling is strongest. Moreover, the unparalleled flexibility of RHIC provides collisions between a variety of different ion species over a broad range in energy. The combined programs permit a comprehensive exploration of the QCD phase diagram, together with detailed studies of how initial conditions affect the creation and dynamical expansion of hot QCD matter and of the microscopic structure of the strongly coupled QGP liquid.

This varied program at these two colliders, covering three orders of magnitude in center of mass energy, has already led to an array of paradigmatic discoveries. Asymmetric \CuAu\ and \HeAu\ collisions\cite{Huang:2012sc}, and collisions between deformed uranium nuclei at RHIC\cite{Wang:2014qxa} are helping to constrain the initial fluctuation spectrum and to eliminate some initial energy deposition models. The recently discovered unexpected collectivity of anisotropic flow signatures observed in
\pPb\ collisions at the LHC\cite{Abelev:2012ola,Aad:2013fja} suggests that similar signatures seen in very-high-multiplicity \pp\ collisions at the LHC\cite{Khachatryan:2010gv} and in a recent re-analysis of 
\dAu\ collisions at RHIC\cite{Adare:2013piz} might also be of collective origin. How collectivity develops in such small systems cries out for explanation. 
The unavoidable question ``What is the smallest size and density of a droplet of QCD matter that behaves like a liquid?'' can only be answered by systematically exploiting RHIC's flexibility to collide atomic nuclei of any size over a wide range of energies.
The additional running in 2015 of \pAu\ and \pAl\ collisions at RHIC will augment the \dAu\ and \HeAu\ data and represents 
an important first step in this direction.

Future precision measurements made possible by increases in the energy and luminosity of the LHC will quantify thermodynamic and microscopic properties of the strongly coupled plasma at temperatures well above the transition temperature $T_C$. At the same time, RHIC  will be the only facility capable of both of providing the experimental lever arm needed to establish the temperature dependence of these parameters while also extending present knowledge of the properties of deconfined matter to larger values of the baryon chemical potential $\mu_B$, where a critical point and first
order phase transition may be awaiting discovery. There is no single facility in the short- or long-term future that could come close to duplicating what RHIC and the LHC, operating in concert, will teach us about Nature.

Both RHIC and the LHC are also capable of probing new, unmeasured physics phenomena at low longitudinal momentum fraction $x$. Proton-lead collisions at the LHC allow the study of previously unreachable regions of phase space in the search for parton saturation effects. However, a complete exploration of parton dynamics at low $x$ will require an Electron-Ion Collider (EIC). 
Accordingly, a cost-effective plan has been developed for a future transition of the RHIC facility to an EIC. 
While forward rapidity studies in \pA\ and \AplusA\ collisions at RHIC and the LHC can provide access to low-$x$ physics in a complementary kinematic range,
an EIC will be needed to deliver crucially missing precise information on the nuclear parton distribution functions within the most desirable kinematic regime. However, this white paper focuses on the future opportunities in the realm of thermal QCD, noting where relevant the need for our knowledge to be extended by the compelling insights to be derived from a future EIC\cite{Accardi:2012qut}.




The remainder of this white paper presents experimental and theoretical progress since the last Long Range Plan in Section~\ref{Sec:Progress} in order to assess the current status of the field and its facilities. Section~\ref{Sec:Future} then describes future prospects for advancing our understanding of thermal QCD and thereby addressing the deep intellectual questions posed by recent discoveries. 
