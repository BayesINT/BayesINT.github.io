\section*{\center{Acronyms and Abbreviations}}
\addcontentsline{toc}{section}{\protect\numberline{}Acronyms and Abbreviations}% Trick to get starred sections into Table of Contents
\label{Sec:AandA}

Acronyms and abbreviations used in this white paper:
\begin{itemize}

\item[\bf AdS:] Anti de Sitter space
\item[\bf ALICE:] A Large Ion Collider Experiment (at the LHC)
\item[\bf AMPT:] A Multi-Phase Transport (theoretical model)
\item[\bf ASW:] Armesto, Salgado, and Wiedemann (theoretical formalism)
\item[\bf ATLAS:] A Toroidal LHC ApparatuS (experiment at the LHC)
\item[\bf BES:] Beam Energy Scan (at RHIC)
\item[\bf BNL:] Brookhaven National Laboratory
\item[\bf BT:] Braaten-Thoma (as used in Figure~\ref{Fig:non-photonic-suppression})
\item[\bf BW:] Brookhaven-Wuppertal (theoretical collaboration)
\item[\bf CERN:] Historical acronym in current use for European Organization for Nuclear Research
\item[\bf CFT:] Conformal Field Theory
\item[\bf CGC:] Color Glass Condensate
\item[\bf CL:] Confidence Limit
\item[\bf CME:] Chiral Magnetic Effect
\item[\bf CMS:] Compact Muon System (experiment at the LHC)
\item[\bf CNM:] Cold Nuclear Matter
\item[\bf CUJET:] Columbia University Jet and Electromagnetic Tomography (theoretical formalism)
\item[\bf CVE:] Chiral Vortical Effect
\item[\bf DOE:] Department of Energy
\item[\bf EIC:] Electron Ion Collider
\item[\bf EM:] ElectroMagnetic (radiation)
\item[\bf EMCAL:] ElectroMagnetic CALorimeter
\item[\bf EOS:] Equation Of State
\item[\bf EPD:] Event Plane Detector (STAR)
\item[\bf FCS:] Forward Calorimetric System (STAR)
\item[\bf FF:] Fragmentation Function
\item[\bf FRIB:] Facility for Rare Isotope Beams
\item[\bf FTS:] Forward Tracking System (STAR)
\item[\bf GEM:] Gas Electron Multiplier
\item[\bf GLV:] Gyulassy-Levai-Vitev (theoretical formalism)
\item[\bf HFT:] Heavy Flavor Tracker (STAR)
\item[\bf HQ:] Heavy Quark
\item[\bf HT-BW:] Higher Twist Berkeley-Wuhan (theoretical formalism)
\item[\bf HT-M:] Higher Twist Majumder (theoretical formalism)
\item[\bf IP-Glasma:] Impact Parameter dependent saturation - color Glass condensate plasma
\item[\bf iTPC:] Inner TPC (STAR)
\item[\bf JET:] Jet and Electromagnetic Tomography (theory-experiment topical group)
\item[\bf JEWEL:] Jet Evolution With Energy Loss (Monte Carlo model)
\item[\bf LANL:] Los Alamos National Laboratory
\item[\bf LHC:] Large Hadron Collider
\item[\bf LO:] Leading Order (in QCD)
\item[\bf L1:] Level 1 (experimental trigger level)
\item[\bf LS1:] Long Shutdown 1 (LHC)
\item[\bf LS2:] Long Shutdown 2 (LHC)
\item[\bf MADAI:] Models and Data Initiative (experimental/ theoretical collaboration)
\item[\bf MARTINI:] Modular Algorithm for Relativistic Treatment of heavy IoN Interactions (Monte Carlo model)
\item[\bf MIE:] Major Item of Equipment (DOE)
\item[\bf MTD:] Muon Telescope Detector (STAR)
\item[\bf MUSIC:] Monotonic Upstream-centered Scheme for Conservation laws for Ion Collisions
\item[\bf NASA:] National Aeronautics and Space Administration
\item[\bf NLO:] Next to Leading Order (in QCD)
\item[\bf PHENIX:]  Pioneering High Energy Nuclear Interaction experiment (at RHIC)
\item[\bf pQCD:] perturbative QCD
\item[\bf PS:] Proton Synchrotron (CERN)
\item[\bf PYQUEN:] PYthia QUENched (Monte Carlo model)
\item[\bf QCD:] Quantum Chromodynamics
\item[\bf QM:] Qin-Majumder (as used in Figure~\ref{Fig:non-photonic-suppression})
\item[\bf RHIC:] Relativistic Heavy Ion Collider
\item[\bf SCET:] Soft Collinear Effective Theory 
\item[\bf sPHENIX:] Super PHENIX
\item[\bf SLAC:] Stanford Linear ACcelerator 
\item[\bf SM:] Standard Model
\item[\bf SPS:] Super Proton Synchrotron (CERN)
\item[\bf STAR:]  Solenoidal Tracker at RHIC
\item[\bf TAMU:] Texas A\&M University
\item[\bf TECHQM:] Theory meets Experiment Collaboration in Hot QCD Matter
\item[\bf TG:] Thoma-Gyulassy (as used in Figure~\ref{Fig:non-photonic-suppression})
\item[\bf TPC:] Time Projection Chamber
\item[\bf VTX:] (silicon) VerTeX tracker (PHENIX)
\item[\bf YaJEM:] Yet another Jet Energy-loss Model
\item[\bf WHDG:]  Wicks, Horowitz, Djordjevic, and Gyulassy (theoretical formalism)

\end{itemize}