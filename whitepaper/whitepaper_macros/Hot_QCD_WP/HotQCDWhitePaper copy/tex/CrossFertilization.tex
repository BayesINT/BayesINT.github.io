\subsection{Broader Impacts}
\label{Sec:Cross}
 
 The intellectual challenges posed by strongly interacting QCD matter have led to significant cross-fertilization with other fields. Theoretical tools have been both imported from and in some cases exported to fields ranging from condensed matter physics to string theory. 
 Recently results from RHIC experimentalists have been announced that have implications beyond QCD at high energy and densities. Of particular note are the ``dark photon" upper limits and the first observation of anti-$^{4}$He. 
It is worth noting that neither of these results were envisioned as part of the experimental program; 
rather they result from the very large data samples that have been acquired at RHIC in combination with exceptional experimental sensitivities\footnote{Essentially identical statements apply to the LHC detectors.}.
Also of interest is a new study in condensed matter physics of the Chiral Magnetic Effect discussed in Section~\ref{Sec:Exotica}.

\subsubsection{Dark Photons}
It has been postulated that an additional U(1) gauge boson, a``dark photon" $U$ that is weakly coupled to ordinary photons, can explain the anomalous magnetic moment of the muon $(g-2)_\mu$, which deviates from  standard model calculations by 3.6$\sigma$. By studying  $\pi^0, \eta \rightarrow e^+e^-$ decays the PHENIX experiment has extracted upper limits on U-$\gamma$ mixing at 90\% CL for decays to known particles, in the mass range 30$<m_U<$90 MeV/$c^2$\cite{Adare:2014mgk}. These results show that except for the small range 30$<m_U<$32 MeV/$c^2$ the 
 U-$\gamma$ mixing parameter space that can explain the $(g-2)_\mu$ deviation from its Standard Model value is  excluded at the 90\% confidence level.
When combined with 
experimental limits from BaBar\cite{Lees:2014xha} and and NA48/2\cite{Adlarson:2014hka}, these analyses essentially
exclude the simplest model of the dark photon as an explanation of the
$(g-2)_\mu$ anomaly.



\subsubsection{Antimatter Nuclei}
In top energy RHIC collisions matter and antimatter are formed with approximately equal rates. The rapid expansion and cooling of the system means that the antimatter decouples quickly from the matter making these types of collisions ideal for studying the formation of anti-nuclei. 
The STAR collaboration has reported detection of anti-helium 4 nuclei ${}^4\mathrm{\overline{He}}$, which is the heaviest anti-nucleus observed to date\cite{Agakishiev:2011ib}. 
The ${}^4\mathrm{\overline{He}}$ yield is consistent with expectations from thermodynamic and coalescent nucleosynthesis models, providing a suggestion
that the detection of even heavier antimatter nuclei is experimentally feasible. These measurements may serve as a benchmark for possible future observations from antimatter sources in the universe.


\subsubsection{Chiral Magnetic Effect in Condensed Matter Systems}
Recently the Chiral Magnetic Effect discussed in Section~\ref{Sec:Exotica} has been observed in a condensed matter experiment measuring magneto-transport in $\mathrm{ZrTe_5}$\cite{Li:2014bha}. 
The recent discovery of Dirac semi-metals 
with chiral quasi-particles\cite{Bor:2014aa,Neu:2014aa,Liu:2014aa} 
has created the opportunity to 
study the CME in condensed matter experiments. 
In these materials it is possible to generate a chiral charge density by the application 
of parallel electric and magnetic fields\cite{Fukushima:2008xe}.
The resulting chiral current is in turn proportional to the product of the 
chiral chemical potential and the magnetic field, leading to a quadratic dependence on the magnetic field.
This is precisely what is observed in Ref.~\cite{Li:2014bha},
where the magnetoconductance varies as the square of the applied magnetic field.
These studies can be extended to a broad range of materials, since three-dimensional 
Dirac semimetals often emerge at quantum transitions between normal and
topological insulators.
Interestingly, the qualitative features observed in \cite{Li:2014bha} 
have been reproduced in a calculation connecting chiral anomalies in 
hydrodynamics with its holographic system in the gauge/gravity duality\cite{Landsteiner:2014vua}. 