\subsection{Towards Quantitative Understanding: Opportunities and Challenges in Theory}
\label{Sec:Theory}

Part of Recommendation IV of the 2007 Long Range Plan was the appreciation that 
{\em ``achieving a quantitative understanding of the properties of the quark-gluon plasma 
also requires new investments in modeling of heavy-ion collisions, in analytical approaches, and in large-scale computing.''} Since then there has been tremendous progress along these lines.
A large and diverse worldwide theory community is working on the challenges
posed by the discoveries of the experimental heavy ion programs at RHIC and at the LHC.
This community develops theoretical tools suited for the upcoming era of detailed experimental
investigation. It includes, amongst others, lattice QCD groups producing
{\em ab initio} calculations of QCD thermodynamics
at  finite temperature and density, nuclear and high energy physicists 
working on the embedding of hard partonic processes in a dense nuclear environment, groups advancing
the development of relativistic fluid dynamic simulations of heavy ion collisions and the interfacing of these
simulations with hadronic cascades, field theorists aiming at developing a description from first principles of
the initial conditions of high parton density and their (non-equilibrium) evolution, 
as well as people developing many-body approaches to evaluate spectral and transport properties of QCD matter,
and string theorists contributing
to the exploration of novel strong coupling techniques suited for the description of strongly coupled, 
nearly perfectly liquid, non-abelian plasmas. 
%This diverse theory community shows all the hallmarks of an active, forward looking
This diverse theory community is active and forward looking;
%community 
%centered around a mature 
it supports, advances and motivates a multifaceted
experimental program, and does so 
with a long perspective. It develops improved 
phenomenological tools that address with increased precision and broadened versatility the diverse needs of 
%a multi-faceted 
the experimental program and mediates its impacts that branch out into neighboring fields of theoretical physics,
including high energy physics, string theory, condensed matter physics 
and astrophysics/cosmology. Here, we highlight only a few 
important recent developments that support these general statements:

\begin{itemize}

\item
Convergence has been reached in lattice QCD calculations of the temperature for the crossover 
transition in strongly interacting matter which has now been established 
at $145\,\mathrm{MeV}{\,<\,}T_c{\,<\,}163\,\mathrm{MeV}]$~\cite{Aoki:2006br,Aoki:2009sc,Bazavov:2011nk,Bazavov:2014pvz}. Continuum extrapolated results for the equation of state, the speed of sound and many other properties 
of strong interaction matter have also been provided \cite{Borsanyi:2013bia,Bazavov:2014pvz}.

\item 
The modeling of the space-time evolution of heavy-ion collisions has become increasingly reliable. (2+1)-dimensional, and subsequently, (3+1)-dimensional relativistic viscous fluid dynamics computations have been performed. 
All such computations use an equation of state extracted from lattice QCD.
Viscous relativistic fluid dynamic (3+1)-dimensional
simulation tools have been developed and subjected to a broad set of 
theoretical precision tests. These tools are instrumental in the ongoing program of extracting 
material properties of the produced  
quark-gluon plasma
from the experimentally observed flow harmonics and reaction plane correlations. 
Within the last five years, in a community-wide effort coordinated by, amongst others,
the TECHQM\cite{TECHQM} initiative, these simulation codes were validated against each other. The range of
applicability of these simulations continues to be pushed to further classes of experimental observables.
Still, already with the limited data/theory comparison tools that have so far been brought to bear on the large sets of experimental data collected at RHIC and LHC, the specific shear viscosity $\eta/s$ of QCD matter created at RHIC could be constrained to be approximately 50\% larger than the limiting value $1/4\pi=0.08$ obtained in strongly coupled plasmas
with a dual gravitational description~\cite{Kovtun:2004de}, 
and to be about 2.5 times larger than this value at the LHC, see e.g.~Ref.~\cite{Gale:2012rq}. 

\item 
The JET collaboration\cite{JET} has coordinated a similarly broad cross-evaluation of the tools available
for the description of jet quenching in hadron spectra,
%Under the aegis of the JET topical collaboration, 
%A successful effort was undertaken to 
and have undertaken a
consolidation of the results of different approaches to 
determining the transport properties of jets as they traverse the strongly correlated quark-gluon plasma.
%Concurrent efforts on extracting the jet quenching parameter $\hat{q}/T^3$ from similar theory-data comparisons have narrowed the 
The range of values for the jet quenching parameter $\hat{q}/T^3$ obtained from  
theory-data comparisons has been narrowed to
%range of values for this parameter to
$2{\,<\,}\hat{q}/T^3{\,<\,}6$ within the temperature range probed by RHIC and the LHC~\cite{Burke:2013yra}.
%, nearly an order of magnitude lower than some previous estimates for this quantity. 
At the same time, a significant number of 
tools were developed for the simulation of full medium-modified parton showers suited for the modeling
of reconstructed jets. In the coming years, these tools will be the basis for a detailed analysis of jet-medium
interactions.  

\item 
There is a community-wide effort devoted to extending CGC calculations from LO
 (current phenomenology) to 
NLO~\cite{Balitsky:2008zza,Chirilli:2011km,Stasto:2013cha,Beuf:2014uia,Kang:2014lha}.
Doing NLO calculations in the presence of a non-perturbatively large parton density requires overcoming
qualitatively novel, conceptually challenging issues that are not present in standard {\em in vacuo} NLO calculations in QCD. Within the last
year, the key issues in this program have been addressed by different groups in independent but consistent
approaches, and the field is now rapidly  advancing these calculations of higher precision to a practically
usable level. 

\item 
In recent years, there have been significant advances in understanding how thermalization occurs in the
initially overoccupied and strongly expanding systems created in heavy ion 
collisions~\cite{Berges:2013eia,Gelis:2013rba,Kurkela:2014tea}. While some of these
developments are still on a conceptual field theoretical level, there is by now the exciting realization that the thermalization
processes identified in these studies share many commonalities with the problem of dynamically describing the
quenching of jets in dense plasmas~\cite{Blaizot:2013hx,Kurkela:2014tla}. 
This is likely to open new possibilities for understanding via the detailed measurements
of jet quenching how non-abelian equilibration processes occur in primordial plasmas. 

\item 
Systematic efforts are being pursued to unravel key properties of QCD matter with heavy-flavor particles. The 
construction of heavy-quark effective theories benefits from increasingly precise information from thermal lattice QCD, to 
evaluate dynamical quantities suitable for phenomenology in heavy-ion collisions (heavy-flavor diffusion coefficient, 
quarkonium spectral properties). This will enable precision tests of low momentum heavy-flavor observables, providing a 
unique window on how in-medium QCD forces vary with temperature.

\item
Much progress has been made towards a systematic understanding from first principles of the 
properties of strongly interacting matter at non-zero baryon number density. Such studies 
rely heavily on the development of theoretical concepts on critical behavior 
signaled by conserved charge fluctuation~\cite{Stephanov:1998dy,Ejiri:2005wq,Stephanov:2011pb}. 
They are accessible to lattice QCD calculations which opens up the possibility, via dynamical modeling, 
for a systematic comparison of experimental fluctuation observables with calculations performed in 
QCD~\cite{Karsch:2012wm,Bazavov:2012vg,Mukherjee:2013lsa,Borsanyi:2013hza,Borsanyi:2014ewa}. This 
will greatly profit from the steady development of computational facilities which are soon expected to deliver 
sustained petaflop/s performance for lattice QCD calculations.

\end{itemize}

%\subsubsection{Open questions and future goals}
%\label{sec:open}

The significant advances listed above document how theory addresses the challenge of keeping pace with the
experimental development towards more complete and more precise exploration of the hot and dense rapidly 
evolving systems produced in heavy ion collisions. We emphasize that all these research directions show strong 
potential for further theoretical development and improved interfacing with future experimental analyses.
Some of the challenging issues over which we need to get better control include: 
i) the pre-equilibrium ``glasma" 
dynamics of coherent gluon fields, and the approach to thermalization; 
ii) the extraction of the values and temperature dependences of transport parameters that reflect the many-body QCD dynamics in deconfined matter; 
iii) the initial conditions at lower collision energies where the Glasma framework breaks down; 
iv) the proper inclusion of the physics of hydrodynamic fluctuations; 
v) an improved treatment of hadron freeze-out and the transition from hydrodynamics  to transport theory, in particular the treatment of viscous corrections that can influence the extraction from data of the physics during the earlier collision stages. 
Quantitative improvements in these aspects of the dynamical modeling of a heavy-ion collision will lead to increased precision in the extraction of the underlying many-body QCD physics that governs the various collision stages. Additional conceptual advances in our understanding of QCD in matter at extreme temperatures and densities are required to answer a number of further outstanding questions. We  list a few of them:

%, in chronological order as seen by an observer inside a heavy-ion collision: 
 
\begin{itemize}

\item
A complete quantitative understanding of the properties of the nuclear wave functions that are resolved in nucleus-nucleus and proton-nucleus collisions remains elusive to date. Progress requires the extension of computations of the energy evolution of these wave functions in the Color Glass Condensate (CGC) framework to next-to-leading logarithmic accuracy as described above, matching these to next-to-leading order perturbative QCD computations at large momenta, and pushing the development of these
calculations into predictive tools. 
Simultaneously, conceptual questions regarding the factorization and universality of distributions need to be addressed for quantitative progress. These ideas will be tested in upcoming proton-nucleus collisions at RHIC and the LHC, and with high precision at a future EIC. 

\item
How the glasma thermalizes to the quark-gluon plasma is not well understood. There has been significant progress in employing classical statistical methods and kinetic theory to the early stage dynamics --- however, these rely on extrapolations of weak coupling dynamics to realistically strong couplings. Significant insight is also provided from extrapolations in the other direction --- from large couplings -- using the holographic AdS/CFT correspondence between strongly coupled ${\cal N}{\,=\,}4$ supersymmetric Yang-Mills theory in four dimensions and weakly coupled gravity in an AdS$_5{\times}$S$_5$ space. Significant numerical and analytical progress can be anticipated in this fast evolving field of non-equilibrium non-Abelian plasmas, with progress on
the question of how
characteristic features of thermalization processes can be constrained in an interplay between experimental and
 theoretical developments. 

\item
A novel development in recent years has been the theoretical study of the possible role of quantum anomalies in heavy-ion collisions. A particular example is the Chiral Magnetic Effect (CME), which explores the phenomenological consequences of topological transitions in the large magnetic fields created at early times in heavy-ion collisions. How the sphaleron transitions that generate topological charge occur out of equilibrium is an outstanding question that can be addressed by both weak coupling and holographic methods. Further, the effects of these charges can be propagated to late times via anomalous hydrodynamics. While there have been hints of the CME in experiments, conventional explanations of these data exist as well. For the future beam energy scan at RHIC, quantifying the predictions regarding signatures of quantum anomalies is crucial. This requires inclusion of the anomalies into the standard 
hydrodynamical framework. We note that the study of the CME has strong cross-disciplinary appeal, with applications in a number of strongly correlated condensed matter systems. 

%\item
%Noteworthy progress has been made in thermal field theory computations of photon and di-lepton production in heavy-ion collisions, where NLO computations are now available. A challenging problem is to find clear signatures of chiral symmetry restoration that are separable from the underlying resonance background. Discrepancies between theory and experiment, for example the ``photon $v_2$ puzzle'' mentioned in Section~\ref{Sec:EM}, point to missing physics, with several unconventional explanations ranging from transient Bose-Einstein Condensates to effects arising from the coupling of the conformal anomaly to external magnetic fields. 

\item
As observed in Section~\ref{Sec:HardProbes}, progress has been made in quantifying the jet quenching parameter $\hat{q}$, which characterizes an important feature of the transverse response of the quark-gluon medium. However, significant challenges persist.  Another important transport parameter $\hat{e}$, characterizing the longitudinal drag of the medium on the hard probe, also needs to be quantified. Much recent theoretical effort has gone into extending the splitting kernel for gluon radiation by a hard parton traversing a dense medium to next-to-leading-order accuracy. In this context Soft Collinear Effective Theory (SCET), imported from high energy theory, has proven a promising theoretical tool whose potential needs to be further explored. There have been recent theoretical developments in understanding how parton showers develop in the quark-gluon medium; confronting these with the available jet fragmentation data requires their implementation in Monte-Carlo codes coupled to a dynamically evolving medium, 
and
the community-wide validation of  these jet quenching 
event generators.
There have been recent attempts to compute the jet quenching parameter using lattice techniques; while very challenging, such studies provide a novel direction to extract information on the non-perturbative dynamics of the strongly correlated quark-gluon plasma.

\item
Quarkonia and heavy flavor, like jets, are hard probes that provide essential information on the quark-gluon plasma on varied length scales. Further, the two probes find common ground in studies of b-tagged and c-tagged jets. In proton-proton and proton-nucleus collisions, non-relativistic-QCD (NRQCD) computations are now standard, and these have been extended to nucleus-nucleus collisions, even to next-to-leading order accuracy. Lattice studies extracting quarkonium and heavy-light meson spectral functions have increased in sophistication, and clear predictions for the sequential melting of quarkonium states exist and need to be confronted with experiment. The direct connection to experiment requires, however, considerable dynamical modeling effort. 
For instance, the question of how the fluid dynamic evolution can be
interfaced with microscopic probes that are not part of the fluid, such as charm and beauty quarks or jets, and
how the yield of electromagnetic processes can be determined with satisfactory precision within this framework, 
are questions of high phenomenological relevance for the experimental program in the coming years, and
the community is turning now to them. 

\item
An outstanding intellectual challenge in the field is to map out the QCD phase diagram. 
We have described the path toward this goal that starts from experimental measurements made
in a beam energy scan in Section~\ref{Sec:CP}.
While the lattice offers an {\it ab initio} approach, its successful implementation is beset by the well known sign problem, which is also experienced in other branches of physics. Nonetheless, approaches employing reweighting and Taylor expansion techniques have become more advanced and are now able to explore the equation of state and freeze-out 
conditions at baryon chemical potentials $\mu_B/T\le 2$. This covers a large part of the energy range currently explored in the 
RHIC Beam Energy Scan and suggests that a possible critical endpoint may only be found at beam energies less than 20~GeV. 
Other promising approaches include the complex Langevin approach \cite{Aarts:2009uq,Aarts:2014kja} and the integration over a Lefschetz thimble \cite{Cristoforetti:2012su,Aarts:2014nxa}. There has been considerable work outlining the phenomenological consequences of a critical point in the phase diagram. However, quantitative modeling of how critical fluctuations affect the measured values of the relevant observables will require the concerted theoretical effort sketched in Section~\ref{Sec:CP}.

\end{itemize}
 
 


%Similar comments apply to 
%the community-wide validation of jet quenching 
%event generators, 

%to the necessity of pushing NLO CGC calculations to predictive tools, 

%or to the question of how
%characteristic features of thermalization processes can be constrained in an interplay between experiment and
%further theoretical development. 

Continued support of these theory initiatives is needed to optimally exploit the
opportunities arising from the continued experimental analysis of heavy ion collisions 
and to interface the insights so obtained with the widest possible cross-section of
the worldwide physics community.
%We cannot emphasize strongly enough that 
Achieving the impressive intellectual achievements we have 
outlined, and meeting the challenges ahead, depend strongly on the development of the theory of 
strongly interacting matter which involves advances in heavy ion phenomenology, 
perturbative QCD, lattice QCD, holographic calculations of equilibration in 
strongly coupled systems, and effective field theories for QCD as well as the strong synergy with 
overlapping and related areas in high energy physics, condensed matter physics, cold atom physics, string theory and studies of complex dynamical systems. 
%In the case of string theory and condensed matter physics, a strong argument can be made that developments in heavy-ion collision theory have influenced developments in those fields. 

Continued advances will also require significantly increased computational resources. 
While lattice QCD continues to play a crucial role by delivering first-principles answers to important questions that require a non-perturbative approach, it does not yet have the ability to address dynamic systems.
Questions such as how the matter formed in heavy ion collisions reaches local thermal equilibrium, 
and how it subsequently evolves to the final state observed in experiments
require sophisticated frameworks that incorporate realistic initial conditions, the interactions of hard processes with the medium, 
and 3+1 viscous hydrodynamics coupled to hadronic transport codes.
Over the last few years, 
the community has developed an arsenal of highly sophisticated dynamical evolution codes 
that simulate the underlying physical mechanisms with unprecedented accuracy 
to provide quantitative predictions for all experimentally accessible observables, 
but at the expense of a huge numerical effort.
In most cases the limiting factor in comparing the output of such calculations to the data
is the accuracy obtainable with the currrently available computational resources rather than the statistical precision of the experimental data. 
Addressing this requires new investments, especially in capacity computing that supplement 
the necessary expansion and upgrades of leadership-class computing facilities.
Specific information on the needed resources for both lattice QCD and dynamical modeling may be found in the report of the Computational Nuclear Physics Meeting Writing Committee\cite{CompReport}, whose 
recommendations we fully endorse.


           


